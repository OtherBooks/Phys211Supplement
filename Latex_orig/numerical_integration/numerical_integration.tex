\chapter[Numerical Iteration]{Solving Equations of Motion Using 
Numerical Iteration}
\label{chapter:numerical}

%\section*{Objectives}
%\begin{objectives}
%\item Know what is meant by a numerical iteration method for solving
%  an equation of motion.
%\item Given the initial position and velocity of a particle acted on
%  by a force in one dimension, calculate the position and velocity
%  several time increments later.
%\end{objectives}

\section{Introduction}

The past few decades have witnessed a massive revolution in the way
people live and work, due in great part to {\em significant} enhancements
in computational power.  Computers are everywhere these days in society,
not just on your desktop (or on your lap) but also in your pockets
(MP3 players and cell phones), in your kitchens (ranges, dishwashers
and microwave ovens), and behind the scenes monitoring the money in
your bank accounts, your class schedules and grades, and your music
preferences at on-line music stores.  

The significant enhancement in
computation power has also dramatically changed all fields of science
and engineering.  Despite our brilliant teaching of physics
in this course, there are many problems in physics and engineering that
you simply will not be able to solve analytically.\footnote{By ``analytically''
we mean using the tools from mathematics to determine a written solution
in the form of an equation that can be used to describe the behavior of the
system.}  Some problems simply don't allow a closed-form solution.
But it is even more severe than that.  There are a wide variety of
physical systems whose equations of motion {\bf can't} be solved, 
no matter how brilliant or persistent the 
scientist/mathematician.  In fact, many real systems are ``chaotic,''
with surprisingly complicated behavior arising from seemingly simple
systems.  In cases where an analytical solution is unavailable, the 
only option is to solve the problem {\em numerically}, using a 
computer to simulate the behavior.

Computer simulations have become among the most important techniques
in science and engineering.  Many of you will use numerical techniques
in your career, whether you are simulating the behavior of a new
passenger airline that you are designing, calculating the forces acting
on an artificial joint that you are designing for a patient, or predicting
the effects of a disruption in Middle East oil supply on the global
economy.  Numerical simulations also play a significant role in basic
scientific research, enabling us to explore the behavior of a system that
is too complicated to solve analytically and to difficult to
explore experimentally.  In fact, numerical simulations are so
common now that they are often considered to be a third branch in
scientific analysis, separate from (and complementary to) experimental
and theoretical science.

The basic idea of numerical simulations is actually quite easy.  
In this chapter, we introduce an important technique referred to as 
{\em iteration} where we break the dynamics of the system into 
a series of discrete time steps.  So, for example, instead of 
representing the motion of a ball with a continuous equation, we
instead note the location of the ball, say, every tenth of a second.
Given the location and velocity of the ball at a particular moment
in time, we can predict its location 0.1 s later by using a very 
simple numerical techniques referred to as the
{\em Euler Method}, a technique that conceptually is nothing more
than a simple application of the common ``distance = speed $\times$ time''
approach.  Despite the simplicity of the Euler method, it is a very
powerful method that is used in many numerical applications.  This chapter
introduces the basic ideas (with some homework problems); you will use
the method in lab to simulate the motion of a falling object subject
to air resistance.

\section{Solving Newton's second law analytically}

Newton's second law $\vec{F}_{\rm net} = m\vec{a}$ is a {\em
differential equation}, i.e., an equation that can be written in
terms of derivatives of various quantities.  Ideally, we would like
to ``solve'' this differential equation to 
determine expressions (as a function of time) for the velocity and
position of a particle moving under the influence of the forces.
If the forces exerted on the particle are
all known, then Newton's second law can be rewritten as
\begin{equation}
  a_x = \frac{d^2x}{dt^2} = \frac{F_{{\rm net},x}}{m},
  \label{eq:newtonII}
\end{equation}
where the forces are assumed to be possibly functions of position and
velocity.  Eq.~(\ref{eq:newtonII}) written in that form is known as
the {\em equation of motion} for the system under consideration.
Mathematically one would proceed by integrating Eq.~(\ref{eq:newtonII}) 
to determine the velocity as a function of time
$v_x(t)$ and then integrating once again to obtain the position as a
function of time $x(t)$.  For example we have learned that for a
particle falling from rest from a height $x_0$ under the force of
gravity $F_{\rm net} = mg$, Eq.~(\ref{eq:newtonII}) becomes
\begin{equation}
  \frac{d^2x}{dt^2} = -g,
  \label{eq:eom-freefall}
\end{equation}
and integrating we obtain the following expressions for the velocity
and position:
\begin{equation}
  v_x(t) = -gt \mbox{\hspace{0.25in}and\hspace{0.25in}}x(t) = x_0 - 
  \frac{1}{2}gt^2.
\end{equation}
If you don't understand how we got these expressions, then take the
derivative with respect to time of $x(t)$ to get $v_x(t)$ and then
$v_x(t)$ to get back to Eq.~(\ref{eq:eom-freefall}).

For the example shown above as well as a few other cases, the equation
of motion is relatively straightforward to integrate to get the
analytical functions for velocity and position.  As discussed in the previous
section, though, there are many cases
where the equations of motion are not so easy to integrate and other
means are necessary for determining the position and velocity of the
particle as a function of time.

In the following sections we will develop a set of equations that we
can use to calculate the position and velocity of a particle at
specified time increments $\Delta t$, a technique called {\em
  numerical iteration}.  Although this technique does not give us as a
final result a neat, compact formula for the position and velocity of
the particle into which any value of time can be inserted, it does
allow us to map out the position and velocity of the particle for an
otherwise mathematically intractable problem.

\section{Numerical Stepping Equations}

Let us incorporate the ideas mentioned above into a set of formulas
that we (or better yet, a computer) could use to calculate the
position and velocity of a particle moving under the influence of some
forces.  Call the present time $t$ and the time a little later $t +
\Delta t$.  Let $x(t)$ denote the position of the particle now, then
$x(t + \Delta t)$ denotes the position of the particle a short time
later.  Similarly $v_x(t)$ and $v_x(t + \Delta t)$ represent the present
and slightly later velocities of the particle.  In all of these
expressions, note that $x(t + \Delta t)$ does not mean the quantity
`$x$' times the quantity `$t + \Delta t$' but rather means the value
of the function $x$ evaluated at the time $t + \Delta t$.  This is
standard functional notation used in mathematics.

Recall the definition of velocity as the rate of change of the
position.  Taking $\Delta t$ to be very small in magnitude, we may
approximate this as ``velocity = displacement/time'' and
express the velocity at time $t$ approximately as
\begin{equation}
  v_x(t) \simeq \frac{\Delta x}{\Delta t} = \frac{x(t+\Delta t) - x(t)}
  {\Delta t}.
\end{equation}
As you recall, $\Delta x/\Delta t$ is the definition of the average
velocity, while the instantaneous velocity is actually the derivative
of the position with respect to time.  However, for small enough time
steps, the average velocity is an excellent approximation for the
instantaneous velocity.

Turning the previous expression around, we can write an expression for
the position of the particle at time $t + \Delta t$ in terms of the
position and velocity at time $t$:
\begin{equation}
  x(t+\Delta t) = x(t) + v_x(t)\Delta t.
  \label{eq:x-step}
\end{equation}

Eq.~(\ref{eq:x-step}) says that the position at time $t + \Delta t$ is
the position at time $t$ plus the distance traveled $v_x(t)\Delta t$ by
the particle during the short time interval $\Delta t$.  Notice that
this result is only approximate because the velocity $v_x$ at time $t$
is not necessarily equal to the average velocity during the entire
time interval.  However, if $\Delta t$ is small enough, the
approximation should be quite good.
   
Next we need an expression for incrementing the velocity.  By analogy
with the arguments leading up to Eq.~(\ref{eq:x-step}), we can write
\begin{equation}
  v_x(t+\Delta t) = v_x(t) + a_x(t)\Delta t.
  \label{eq:v-step}
\end{equation}

The three equations (\ref{eq:newtonII}), (\ref{eq:x-step}) and
(\ref{eq:v-step}) can now be incorporated into a looping procedure in
a computer program.  These three equations constitute what is
generally referred to as {\em Euler's method} of numerical
approximation.  Given an initial position and velocity, we calculate
the initial acceleration from Eq.~(\ref{eq:newtonII}).  Then we
calculate the position and velocity a short time later from
Eqs.~(\ref{eq:x-step}) and (\ref{eq:v-step}).  Then we repeat the
process, pretending that the new values for $x$ and $v_x$ are the
initial values.  In this way we can numerically iterate the motion of
the particle from instant to instant as far into the future as we care
to.  A spread-sheet program, such as EXCEL, can perform such
calculations with very little ``programming'' required on your part.

A note about numerical errors is worth mentioning.  Remember that
although Eq.~(\ref{eq:newtonII}) is exact, Eqs.~(\ref{eq:x-step}) and
(\ref{eq:v-step}) that update $x$ and $v_x$ to later times are
approximations that are best when $\Delta t$ is small.  If the
calculations start going haywire, we can help the situation by
choosing smaller steps.  This means of course that the computer will
have to run longer, but that's frequently not a serious problem.

\section{Numerical Solution for a Mass on a Spring}

Let's apply this new method to a system we will be studying more in
depth later in this course.  The system is a mass which moves under
the influence of a force exerted on it by a spring.  The spring is a
device which exerts a force which is proportional to the displacement
of the mass from an equilibrium position.  Taking the equilibrium
position to be $x = 0$, this implies that the acceleration of the mass
is directly proportional to the position $x(t)$.  Suppose in our
particular system the acceleration is given by 
\begin{equation}
a_x(t) = -2.00\, x(t).
\end{equation}
The minus sign in this expression tells us that the force is
always opposite to the displacement.  We'll also assume that time is
in seconds, position is in meters, velocity is in meters per second,
and acceleration is in meters per second squared
   
To proceed, we choose time steps of size $\Delta t = 0.10$ s and start
the clock at $t = 0$.  We could pick any initial position and
velocity; let's choose to release the mass from rest at a position
0.30 m from equilibrium, i.e. $x(0) = 0.30$ m and $v_x(0) = 0$.  Let's
walk through the first few steps and then show some results from a
computer spreadsheet.
   
For our example Eqs.~(\ref{eq:newtonII}), (\ref{eq:x-step}) and
(\ref{eq:v-step}) are written as
\begin{align}
a_x(t) &= -2.00\, x(t) 
\label{eq:ex-a}\\
x(t+0.10) &= x(t) + 0.10\, v_x(t) 
\label{eq:ex-x}\\
v_x(t+0.10) &= v_x(t) + 0.10\, a_x(t).
\label{eq:ex-v}
\end{align}
First calculate the initial acceleration by setting $t = 0$ in
Eq.~(\ref{eq:ex-a}) to find
\begin{equation}
a_x(0) = -2.00\, x(0)  = -2.00 \times 0.30 = -0.60.
\end{equation}
Then update $x(t)$ and $v_x(t)$ by setting $t = 0$ in Eqs.~(\ref{eq:ex-x}) and 
(\ref{eq:ex-v}):
\begin{align}
x(0.10) &= x(0) + 0.10\, v_x(0) = 0.30 + 0.10\times 0 = 0.30 \\
v_x(0.10) &= v_x(0) + 0.10\, a_x(0) = 0 + 0.10\times(-0.60) = -0.06.
\end{align}
Since the mass was initially at rest, a short time later it is still
approximately at the same location.  However, since the spring is
stretched at $t = 0$, a force is acting on the mass immediately, so
that a short time later it has already acquired a non-zero velocity.
   
How would you find $x(0.20)$ and $v_x(0.20)$?  Again use
Eqs.~(\ref{eq:ex-a}) through (\ref{eq:ex-v}), this time with the
`present time' $t = 0.10$.  We find that
\begin{align}
a_x(0.20) &= -2.00\, x(0.10) = -2.00\times 0.30 = -0.60\\
x(0.20) &= x(0.10) + 0.10\, v_x(0.10) \nonumber \\
        &= 0.30 + 0.10 \times (-0.06) \nonumber \\
        &= 0.294 \\
v_x(0.20) &= v_x(0.10) + 0.10\, a_x(0.10) \nonumber  \\
        &= -0.06 + 0.10\times(-0.60) \nonumber \\
        &= = -0.12.
\end{align}
We can continue this process as long as we like.  You will find it
convenient to organize the information for the position, velocity and
acceleration for each time in the form of a table.  Table
\ref{table:numerical1} on the next page lists $t$, $x$, $v_x$ and $a_x$ for
the motion of this mass.  Note that the periodic nature of the motion
is manifested in the entries of the table.  

There is an unsettling aspect of the entries in Table
\ref{table:numerical1}.  We started at $x = 0.30\, \mbox{m}$, but at
$t = 2.30\, \mbox{s}$ the position of the mass is $x = -0.375\,
\mbox{m}$, and further down in the table we find that at $t = 4.5$ s,
$x = 0.468\, \mbox{m}$.  What should we have expected?  If we had a
real mass connected to a spring and set it oscillating we would expect
the amplitude of the oscillations to gradually decrease because of the
presence of dissipative forces (air resistance and the imperfect
elasticity of the spring).  In an ideal case, with no dissipative
effects, we would expect there to be no increase or decrease in the
amplitude; that is, the mass should oscillate between $x = +0.30\,
\mbox{m}$ and $x = -0.30\, \mbox{m}$.  But this is not the case if we
look at the data in Table \ref{table:numerical1}.  The problem is that
we used too large a time increment.  Why does too large a time
increment lead to errors?  If you recall, our stepping equations use
the approximation that the average velocity is very close to the
instantaneous velocity.  If the time step is too large, this
approximation is no longer valid and leads to errors.
   
We can improve our calculation of the motion by choosing a smaller
time increment $\Delta t$.  If we choose $\Delta t = 0.01\, \mbox{s}$
rather than $0.10\, \mbox{s}$, we would be calculating over a much
finer time interval (10 times smaller) and while we will have to do 10
times more computations to evolve the motion out to the same time, the
calculations should be more accurate.  Table \ref{table:numerical2}
lists $t$, $x$, $v_x$ and $a_x$ near a point of maximum displacement for
this smaller time increment.  The maximum displacement is now about
0.314.  This is still larger than the initial displacement but not
nearly as bad as before.  Further reduction of the time increment
would improve the result.

%%% I had to mess around a lot to get floats to appear correctly.  ML

\addtolength{\textheight}{.5in}
\addtolength{\footskip}{-2.in}
\mbox{}
\vspace{-2.in}

\twocolumn

\begin{table}[!t]
\caption{Numerical solution for motion of 
mass on a spring using $\Delta t=0.10\, \mbox{s}$}
\begin{small}
\begin{center}
\begin{tabular}{cccc}
$t$ & $x(t)$ & $v_x(t)$ & $a_x(t)$ \\
\hline
0	&	0.300	&	0	&	-0.600\\
0.100	&	0.300	&	-0.060	&	-0.600\\
0.200	&	0.294	&	-0.120	&	-0.588\\
0.300	&	0.282	&	-0.179	&	-0.564\\
0.400	&	0.264	&	-0.235	&	-0.528\\
0.500	&	0.241	&	-0.288	&	-0.481\\
0.600	&	0.212	&	-0.336	&	-0.424\\
0.700	&	0.178	&	-0.379	&	-0.356\\
0.800	&	0.140	&	-0.414	&	-0.281\\
0.900	&	0.099	&	-0.442	&	-0.198\\
1.000	&	0.055	&	-0.462	&	-0.109\\
1.100	&	0.008	&	-0.473	&	-0.017\\
1.200	&	-0.039	&	-0.475	&	0.078\\
1.300	&	-0.086	&	-0.467	&	0.173\\
1.400	&	-0.133	&	-0.450	&	0.266\\
1.500	&	-0.178	&	-0.423	&	0.356\\
1.600	&	-0.220	&	-0.387	&	0.440\\
1.700	&	-0.259	&	-0.343	&	0.518\\
1.800	&	-0.293	&	-0.292	&	0.587\\
1.900	&	-0.322	&	-0.233	&	0.645\\
2.000	&	-0.346	&	-0.168	&	0.692\\
2.100	&	-0.363	&	-0.099	&	0.725\\
2.200	&	-0.373	&	-0.027	&	0.745\\
2.300	&	-0.375	&	0.048	&	0.750\\
2.400	&	-0.370	&	0.123	&	0.741\\
2.500	&	-0.358	&	0.197	&	0.716\\
2.600	&	-0.338	&	0.268	&	0.677\\
2.700	&	-0.312	&	0.336	&	0.623\\
2.800	&	-0.278	&	0.398	&	0.556\\
2.900	&	-0.238	&	0.454	&	0.476\\
3.000	&	-0.193	&	0.502	&	0.386\\
3.100	&	-0.143	&	0.540	&	0.285\\
3.200	&	-0.089	&	0.569	&	0.177\\
3.300	&	-0.032	&	0.587	&	0.063\\
3.400	&	0.027	&	0.593	&	-0.054\\
3.500	&	0.086	&	0.587	&	-0.173\\
3.600	&	0.145	&	0.570	&	-0.290\\
3.700	&	0.202	&	0.541	&	-0.404\\
3.800	&	0.256	&	0.501	&	-0.512\\
3.900	&	0.306	&	0.450	&	-0.612\\
4.000	&	0.351	&	0.388	&	-0.702\\
4.100	&	0.390	&	0.318	&	-0.780\\
4.200	&	0.422	&	0.240	&	-0.844\\
4.300	&	0.446	&	0.156	&	-0.892\\
4.400	&	0.461	&	0.067	&	-0.923\\
4.500	&	0.468	&	-0.026	&	-0.936\\
4.600	&	0.465	&	-0.119	&	-0.931\\

\hline
\end{tabular}
\end{center}
\end{small}
\label{table:numerical1}
\end{table}
\newpage

%\mbox{}
\begin{table}[!t]
\caption{Data for mass on a spring near a turning point using  
$\Delta t=0.01\, \mbox{s}$}
\begin{small}
\begin{center}
\begin{tabular}{cccc}
$t$ & $x(t)$ & $v_x(t)$ & $a_x(t)$ \\
\hline
4.190	&	0.293	&	0.155	&	-0.586\\
4.200	&	0.295	&	0.149	&	-0.589\\
4.210	&	0.296	&	0.143	&	-0.592\\
4.220	&	0.297	&	0.137	&	-0.595\\
4.230	&	0.299	&	0.131	&	-0.598\\
4.240	&	0.300	&	0.125	&	-0.600\\
4.250	&	0.301	&	0.119	&	-0.603\\
4.260	&	0.303	&	0.113	&	-0.605\\
4.270	&	0.304	&	0.107	&	-0.607\\
4.280	&	0.305	&	0.101	&	-0.610\\
4.290	&	0.306	&	0.095	&	-0.612\\
4.300	&	0.307	&	0.089	&	-0.614\\
4.310	&	0.308	&	0.083	&	-0.615\\
4.320	&	0.308	&	0.077	&	-0.617\\
4.330	&	0.309	&	0.071	&	-0.619\\
4.340	&	0.310	&	0.064	&	-0.620\\
4.350	&	0.311	&	0.058	&	-0.621\\
4.360	&	0.311	&	0.052	&	-0.622\\
4.370	&	0.312	&	0.046	&	-0.623\\
4.380	&	0.312	&	0.040	&	-0.624\\
4.390	&	0.313	&	0.033	&	-0.625\\
4.400	&	0.313	&	0.027	&	-0.626\\
4.410	&	0.313	&	0.021	&	-0.626\\
4.420	&	0.313	&	0.014	&	-0.627\\
4.430	&	0.314	&	0.008	&	-0.627\\
4.440	&	0.314	&	0.002	&	-0.627\\
4.450	&	0.314	&	-0.004	&	-0.627\\
4.460	&	0.314	&	-0.011	&	-0.627\\
4.470	&	0.313	&	-0.017	&	-0.627\\
4.480	&	0.313	&	-0.023	&	-0.627\\
4.490	&	0.313	&	-0.029	&	-0.626\\
4.500	&	0.313	&	-0.036	&	-0.626\\
4.510	&	0.312	&	-0.042	&	-0.625\\
4.520	&	0.312	&	-0.048	&	-0.624\\
4.530	&	0.312	&	-0.054	&	-0.623\\
4.540	&	0.311	&	-0.061	&	-0.622\\
4.550	&	0.310	&	-0.067	&	-0.621\\
4.560	&	0.310	&	-0.073	&	-0.619\\
4.570	&	0.309	&	-0.079	&	-0.618\\
4.580	&	0.308	&	-0.085	&	-0.616\\
4.590	&	0.307	&	-0.092	&	-0.615\\
4.600	&	0.306	&	-0.098	&	-0.613\\
4.610	&	0.305	&	-0.104	&	-0.611\\
4.620	&	0.304	&	-0.110	&	-0.609\\
4.630	&	0.303	&	-0.116	&	-0.607\\
4.640	&	0.302	&	-0.122	&	-0.604\\
4.650	&	0.301	&	-0.128	&	-0.602\\

\hline
\end{tabular}
\end{center}
\end{small}
\label{table:numerical2}
\end{table}

\onecolumn 
\addtolength{\textheight}{-.5in} 
\addtolength{\footskip}{2.in}



%\addtolength{\textheight}{-2.in}
%\addtolength{\footskip}{2.in}

\newpage

\section*{Problems}

\begin{problem}
  For a certain mass-spring system the acceleration is given by $a_x(t) =
  -0.10\, x(t)$.  Suppose the initial position and velocity are $x(0)
  = 10$ m and $v_x(0) = -1.0\, \mbox{m/s}$.  Calculate $x(t)$ and $v_x(t)$
  at $t = 2\, \mbox{s}$ in two different ways:
  \begin{enumerate}
  \item Use two steps of 1 second each.
  \item Use four steps of $\frac{1}{2}$ second each.  Round only your
    final results to three digits (keep all digits for the
    intermediate calculations).
  \item Why aren't the answers to a) and b) the same?
  \end{enumerate}
  \label{problem:ho-undamped-num}
\end{problem}


\begin{problem}
  A drag force on an object is opposite to its velocity and is often
  proportional to its speed.  Let's immerse the mass-spring system of
  problem (\ref{problem:ho-undamped-num}) in a vat of salad oil so
  that the acceleration becomes
  \[a_x(t) = -0.10\, x(t) - v_x(t)\] 

  Repeat problem \ref{chapter:numerical}.\ref{problem:ho-undamped-num}
  for this acceleration.  Compare the results with those you
  originally got in problem
  \ref{chapter:numerical}.\ref{problem:ho-undamped-num}.  Are the
  results what you might expect when a drag force is present?
\end{problem}
